
\chapter*{Estructura del Trabajo}

    Este trabajo de título se dividirá en los siguientes 4 capítulos, los cuales se describen a continuación:

    \begin{itemize}
      \item \textbf{Capítulo 1: Introducción} 

      En el primer capitulo se explica el problema que se quiere resolver mediante microservicios, además de detallar el estado del arte y los objetivos, tanto de forma general, como específicos.

      \item \textbf{Capítulo 2: Marco Teórico} 

      En este capitulo, se detallan en profundidad todos los aspectos teóricos que implican tanto, el desarrollo de APIs, como de su arquitectura. Interiorizando todos los conceptos y fundamentos, además de las herramientas tecnológicas que se requieren para crear una plataforma de microservicios.

      \item \textbf{Capítulo 3: Metodología y Planificación} 

      El tercer capitulo se explica la metodología Ágil implementada

      \item \textbf{Capítulo 4: Diseño} 
    
      En el cuarto capitulo se define el diseño que se emplea, explicando inicialmente la aplicación de forma general, para enseguida explicar la arquitectura de sus componentes de forma genera y separa por backend y frontend.
      
      \item \textbf{Capítulo 5: Implementación} 

     En el penúltimo capitulo se procede a mostrar la implementación del desarrollo.
     

      \item \textbf{Capítulo 6: Conclusiones y Trabajo a Futuro} 

    Finalmente en el ultimo capitulo, se reflexiona acerca del trabajo propuesto y el trabajo a futuro del proyecto.

    \end{itemize}