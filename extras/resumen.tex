\chapter*{Resumen} % si no queremos que añada la palabra "Capitulo"

El objetivo primordial de este trabajo es implementar una plataforma  de apis centralizada, basada en una arquitectura de micro servicios en la nube. En el presente informe hablaremos de cuales son la  ventajas y desventajas de una arquitectura de micro servicios implementada en ambientes cloud, enfocando además las ventajas de este tipo de implementación, frente a una monolítica. La contruccion de las  APIs implementadas utilizaron como puente de conexión a Node-Red, el cual es una herramienta de programación para conectar APIs y servicios en línea. Este proyecto se realizo bajo la plataforma Google Cloud Plataform(GCP) y serán desplegadas en producción en ambientes contenerizados utilizando Docker y Kubernetes. En resumen trataremos los principales paradigamas en el desarrollo de apis basados en micro servicios en ambientes cloud. 

\hfill

Palabras Clave: APIs, Microservicios, Contenedores, Docker, Kubernetes, Cluster, Node-Red, GCP.

\addcontentsline{toc}{chapter}{Resumen}
\clearpage

\chapter*{Abstract}

The primary goal of this work is to implement a centralized apis platform based on a cloud microservices architecture. In this report we will discuss the advantages and disadvantages of a microservices architecture implemented in cloud environments, focusing also on the advantages of this type of implementation, versus a monolithic one. The construction of the implemented APIs used Node-Red as a connection bridge, which is a programming tool to connect APIs and online services. This project was done under the Google Cloud Platform (GCP) and will be deployed in production in containerized environments using Docker and Kubernetes. In summary we will discuss the main paradigms in the development of apis based on microservices in cloud environments. 

\hfill

Keywords: APIs, Microservices, Containers, Docker, Kubernetes, Cluster, Node-Red, GCP.

\addcontentsline{toc}{chapter}{Abstract}
